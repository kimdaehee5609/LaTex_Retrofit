%	-------------------------------------------------------------------------------
% 
%
%	------------------------------------------------------------------------------- itemize
%
%	------------------------------------------------------------------------------- enumerate
%
%	------------------------------------------------------------------------------- description
%
%	------------------------------------------------------------------------------- parlist
%
%	------------------------------------------------------------------------------- tabbing
%
%	------------------------------------------------------------------------------- tabenum
%
%
%
%
%	-------------------------------------------------------------------------------
%	\documentclass[12pt, a4paper, twoside]{book}
%	\documentclass[12pt, a4paper, twoside, openright]{book}
	\documentclass[12pt, a4paper, oneside]{book}
%	\documentclass[12pt, a4paper, landscape, oneside]{book}

		% --------------------------------- 페이지 스타일 지정
		\usepackage{geometry}
%		\geometry{landscape=true	}
		\geometry{top 		=10em}
		\geometry{bottom		=10em}
		\geometry{left		=8em}
		\geometry{right		=8em}
		\geometry{headheight	=4em} % 머리말 설치 높이
		\geometry{headsep		=2em} % 머리말의 본문과의 띠우기 크기
		\geometry{footskip		=4em} % 꼬리말의 본문과의 띠우기 크기
% 		\geometry{showframe}
	
%		paperwidth 	= left + width + right (1)
%		paperheight 	= top + height + bottom (2)
%		width 		= textwidth (+ marginparsep + marginparwidth) (3)
%		height 		= textheight (+ headheight + headsep + footskip) (4)



		%	===================================================================
		%	package
		%	===================================================================
%			\usepackage[hangul]{kotex}				% 한글 사용
			\usepackage{kotex}						% 한글 사용
			\usepackage[unicode]{hyperref}			% 한글 하이퍼링크 사용
			\usepackage{amssymb,amsfonts,amsmath}	% 수학 수식 사용

			\usepackage{scrextend}					% 
		
		% ------------------------------ 개조식 문서 작성
			\usepackage{enumerate}			%
			\usepackage{enumitem}			%
			\usepackage{tabto}				%     tabto package
			\usepackage{tablists}			%	수학문제의 보기 등을 표현하는데 사용
										%	tabenum




		% ------------------------------ box & minipage
			\usepackage{setspace}			%
			\usepackage{booktabs}			% table
			\usepackage{color}				%
			\usepackage{boxedminipage}		% 미니 페이지
			\usepackage[pdftex]{graphicx}	% 그림 사용
			\usepackage[final]{pdfpages}	% pdf 사용
			\usepackage{framed}			% pdf 사용
			
			\usepackage{fix-cm}	
			\usepackage[english]{babel}
	
			\usepackage{tikz}%
			\usetikzlibrary{arrows,positioning,shapes}
			%\usetikzlibrary{positioning}

		% ------------------------------ 그림
			\usepackage[pdftex]{graphicx} 	% 그림
			\usepackage{floatflt} 			% 그림
			\usepackage{subfigure} 		% 그림
			\usepackage{pdfpages} 			% 그림
			


		%	=======================================================================================
		% 	tikz package
		% 	
		% 	--------------------------------- 	
			\usepackage{tikz}%
			\usetikzlibrary{arrows,positioning,shapes}
			\usetikzlibrary{mindmap,trees}			
		%	=======================================================================================
		% 	mdframed package
		% 	
		% 	--------------------------------- 	
			\usepackage[framemethod=TikZ]{mdframed}				% md framed package
			\usepackage{smartdiagram}							% smart diagram package


		% ------------------------------ table 
			\usepackage{longtable}			%
%			\usepackage[table]{xcolor}		 xcolor 이 충돌을 자꾸 일으킴
			\usepackage{rotating}
			\usepackage{array}
			\usepackage{tabularx}
			\usepackage{tabulary}
			\usepackage{multirow}

			\usepackage{tabu}				%
			\usepackage{tabto}				%  tabto package  
%			\usepackage{subcaption}		%  sub caption package  : 다중 캡션  subfigure와 충돌됨


		% --------------------------------- 	page
			\usepackage{afterpage}			% 다음페이지가 나온면 어떻게 하라는 명령 정의 패키지
%			\usepackage{fullpage}			% 잘못 사용하면 다 흐트러짐 주의해서 사용
%			\usepackage{pdflscape}			% 
			\usepackage{lscape}			%	 


			\usepackage{blindtext}
			\usepackage{lipsum}
	
		% --------------------------------- 특수문자 font 사용
			\usepackage{pifont}			%
			\usepackage{textcomp}
			\usepackage{gensymb}
			\usepackage{marvosym}



			\usepackage[cc]{titlepic}  		% 표지에 그림 넣기 위한 패키지
			\usepackage{listings}			%
			





		% --------------------------------- 페이지 스타일 지정

		\usepackage[Sonny]		{fncychap}

			\makeatletter
			\ChNameVar	{\Large\bf}
			\ChNumVar	{\Huge\bf}
			\ChTitleVar	{\Large\bf}
			\ChRuleWidth	{0.5pt}
			\makeatother

%		\usepackage[Lenny]		{fncychap}
%		\usepackage[Glenn]		{fncychap}
%		\usepackage[Conny]		{fncychap}
%		\usepackage[Rejne]		{fncychap}
%		\usepackage[Bjarne]	{fncychap}
%		\usepackage[Bjornstrup]{fncychap}

		\usepackage{fancyhdr}
		\pagestyle{fancy}
		\fancyhead{} % clear all fields
		\fancyhead[LO]{\footnotesize \leftmark}
		\fancyhead[RE]{\footnotesize \leftmark}
		\fancyfoot{} % clear all fields
		\fancyfoot[LE,RO]{\large \thepage}
		%\fancyfoot[CO,CE]{\empty}
		\renewcommand{\headrulewidth}{1.0pt}
		\renewcommand{\footrulewidth}{0.4pt}
	

		%	--------------------------------------------------------------------------------------- 
		% 	tritlesec package
		% 	
		% 	
		% 	------------------------------------------------------------------ section 스타일 지정
			\usepackage{titlesec}
		
		% 	----------------------------------------------------------------- section 글자 모양 설정
			\titleformat*{\section}					{\large\bfseries}
			\titleformat*{\subsection}				{\normalsize\bfseries}
			\titleformat*{\subsubsection}			{\normalsize\bfseries}
			\titleformat*{\paragraph}				{\normalsize\bfseries}
			\titleformat*{\subparagraph}			{\normalsize\bfseries}
	
		% 	----------------------------------------------------------------- section 번호 설정
			\renewcommand{\thepart}				{\arabic{part}.}
			\renewcommand{\thesection}				{\arabic{section}.}
			\renewcommand{\thesubsection}			{\thesection\arabic{subsection}.}
			\renewcommand{\thesubsubsection}		{\thesubsection\arabic{subsubsection}}
			\renewcommand{\theparagraph} 			{$\blacksquare$ \hspace{3pt}}

		% 	----------------------------------------------------------------- section 페이지 나누기 설정
			\let\stdsection\section
			\renewcommand\section{\newpage\stdsection}

		% --------------------------------- 	장, 절 등의 번호와 장, 절 등과의 간격 조정
		
			\titlespacing*{\section} 				{0pt}{1.0em}{1.0em}
			\titlespacing*{\subsection}	  		{0ex}{1.0em}{1.0em}
			\titlespacing*{\subsubsection}			{0ex}{1.0em}{1.0em}
			\titlespacing*{\paragraph}			{0ex}{1.0em}{1.0em}
			\titlespacing*{\subparagraph}			{0ex}{1.0em}{1.0em}

		% --------------------------------- recommend		섹션별 페이지 상단 여백
			\newcommand{\SectionMargin}			{\newpage  \null \vskip 2cm}
			\newcommand{\SubSectionMargin}		{\newpage  \null \vskip 2cm}
			\newcommand{\SubSubSectionMargin}		{\newpage  \null \vskip 2cm}



	
		% --------------------------------- mini toc 설정
			\usepackage{minitoc}


%		\renewcommand*{\partheadstartvskip}{\null\vskip20pt}
%		\renewcommand*{\partheadendvskip}{\vskip2pt}
%		\renewcommand\beforeparttoc{}		
		
		
		% --------------------------------- 절의 목차
		\setcounter{parttocdepth}{1}
		\setlength{\ptcindent}{0pt}
		\renewcommand{\ptcfont}{\normalsize\rm}
		\renewcommand{\ptcCfont}{\normalsize\bf}
		\renewcommand{\ptcSfont}{\normalsize\rm}
		
		% --------------------------------- 장의 목차
		\setcounter{minitocdepth}{1}    	% Show until subsubsections in minitoc
		\setlength{\mtcindent}{12pt} 		% default 24pt

	
	
		% --------------------------------- 	문서 기본 사항 설정
		\setcounter{secnumdepth}{4} 		% 문단 번호 깊이
		\setcounter{tocdepth}{3} 			% 문단 번호 깊이
		\setlength{\parindent}{0cm} 		% 문서 들여 쓰기를 하지 않는다.
		

		% --------------------------------- 	찾아보기
		\usepackage{makeidx}
		\makeindex

		% --------------------------------- 	줄간격 설정
		\doublespace
%		\onehalfspace
%		\singlespace
		
		
% 	============================================================================== List global setting
%		\setlist{itemsep=1.0em}
	
% 	============================================================================== enumi setting

%		\renewcommand{\labelenumi}{\arabic{enumi}.} 
%		\renewcommand{\labelenumii}{\arabic{enumi}.\arabic{enumii}}
%		\renewcommand{\labelenumii}{(\arabic{enumii})}
%		\renewcommand{\labelenumiii}{\arabic{enumiii})}


	%	-------------------------------------------------------------------------------
	%		Vertical and Horizontal spacing
	%	-------------------------------------------------------------------------------
		\setlist[enumerate,1]	{ leftmargin=8.0em, rightmargin=0.0em, labelwidth=0.0em, labelsep=0.0em }
		\setlist[enumerate,2]	{ leftmargin=8.0em, rightmargin=0.0em, labelwidth=0.0em, labelsep=0.0em }
		\setlist[enumerate,3]	{ leftmargin=8.0em, rightmargin=0.0em, labelwidth=0.0em, labelsep=0.0em }
		\setlist[enumerate]	{ 	itemsep=0.0em, 
								leftmargin=6.0ex, 
								rightmargin=0.0em, 
								labelwidth=0.0em, 
								labelsep=4.0ex 
							}


	%	-------------------------------------------------------------------------------
	%		Label
	%	-------------------------------------------------------------------------------
%		\setlist[enumerate,1]{ label=\arabic*., ref=\arabic* }
%		\setlist[enumerate,1]{ label=\emph{\arabic*.}, ref=\emph{\arabic*} }
%		\setlist[enumerate,1]{ label=\textbf{\arabic*.}, ref=\textbf{\arabic*} }   	% 1.
%		\setlist[enumerate,1]{ label=\textbf{\arabic*)}, ref=\textbf{\arabic*)} }		% 1)
		\setlist[enumerate,1]{ label=\textbf{(\arabic*)}, ref=\textbf{(\arabic*)} }	% (1)
		\setlist[enumerate,2]{ label=\textbf{\arabic*)}, ref=\textbf{\arabic*)} }		% 1)
		\setlist[enumerate,3]{ label=\textbf{\arabic*.}, ref=\textbf{\arabic*.} }		% 1.

%		\setlist[enumerate,2]{ label=\emph{\alph*}),ref=\theenumi.\emph{\alph*} }
%		\setlist[enumerate,3]{ label=\roman*), ref=\theenumii.\roman* }


	% 	============================================================================== itemi Global setting

	
		%	-------------------------------------------------------------------------------
		%		Vertical spacing
		%	-------------------------------------------------------------------------------
			\setlist[itemize]{topsep=0.0em}			% 상단의 여유치
			\setlist[itemize]{partopsep=0.0em}			% 
			\setlist[itemize]{parsep=0.0em}			% 
%			\setlist[itemize]{itemsep=0.0em}			% 
			\setlist[itemize]{noitemsep}				% 
			
		%	-------------------------------------------------------------------------------
		%		Horizontal spacing
		%	-------------------------------------------------------------------------------
			\setlist[itemize]{labelwidth=1em}			%  라벨의 표시 폭
			\setlist[itemize]{leftmargin=8em}			%  본문 까지의 왼쪽 여백  - 4em
			\setlist[itemize]{labelsep=3em} 			%  본문에서 라벨까지의 거리 -  3em
			\setlist[itemize]{rightmargin=0em}			% 오른쪽 여백  - 4em
			\setlist[itemize]{itemindent=0em} 			% 점 내민 거리 label sep 과 같은면 점위치 까지 내민다
			\setlist[itemize]{listparindent=3em}		% 본문 드려쓰기 간격
	
	
			\setlist[itemize]	{ 
						topsep=0.0em, 			%  상단의 여유치
						partopsep=0.0em, 		%  
						parsep=0.0em, 
						itemsep=0.0em, 
						labelwidth=1em, 
						leftmargin=2.5em,
						labelsep=2em,			%  본문에서 라벨 까지의 거리
						rightmargin=0em,		% 오른쪽 여백  - 4em
						itemindent=0em, 		% 점 내민 거리 label sep 과 같은면 점위치 까지 내민다
						listparindent=0em,		% 본문 드려쓰기 간격
						}

			\setlist[description]	{ 
							topsep=0.0em, 			%  상단의 여유치
							partopsep=0.0em, 		%  
							parsep=0.0em, 
							itemsep=0.0em, 
							labelwidth=1em, 
							leftmargin=2em,
							labelsep=1em,			%  본문에서 라벨 까지의 거리
							rightmargin=0em,		% 오른쪽 여백  - 4em
							itemindent=0em, 		% 점 내민 거리 label sep 과 같은면 점위치 까지 내민다
							listparindent=0em,		% 본문 드려쓰기 간격
							}

	%	-------------------------------------------------------------------------------
	%		Label
	%	-------------------------------------------------------------------------------
		\renewcommand{\labelitemi}{$\bullet$}
		\renewcommand{\labelitemii}{$\cdot$}
		\renewcommand{\labelitemiii}{$\diamond$}
		\renewcommand{\labelitemiv}{$\ast$}		




		% --------------------------------- recommend  글자 색깔지정 명령
		\newcommand{\red}		{\color{red}}			% 글자 색깔 지정
		\newcommand{\blue}		{\color{blue}}		% 글자 색깔 지정
		\newcommand{\black}	{\color{black}}		% 글자 색깔 지정
		\newcommand{\superscript}[1]{${}^{#1}$}

	
	
		% --------------------------------- 환경 정의 : 박스 치고 안의 글자 빨간색

			\newenvironment{BoxRedText}
			{ 	\setlength{\fboxsep}{12pt}
				\begin{boxedminipage}[c]{1.0\linewidth}
				\color{red}
			}
			{ 	\end{boxedminipage} 
				\color{black}
			}
			
			
			
		% --------------------------------- 예제를 위한 정의 


			\newenvironment{ex-001}{\textbf{연습문제}\begin{itshape}} {\end{itshape}}

			\newenvironment{env_ex-002}
			{ 	\setlength{\fboxsep}{12pt}
				\begin{boxedminipage}[c]{1.0\linewidth}
				\color{red}
			}
			{ 	\end{boxedminipage} 
				\color{black}
			}

			

% ------------------------------------------------------------------------------
% Begin document (Content goes below)
% ------------------------------------------------------------------------------
	\begin{document}
	
		% ----------------------- 장 목차 생성 Initialization
			\noptcrule 	% part toc에서 줄이 침범하는 에러가 나서 줄긋기를 중지 시킴			
			\dominitoc 		
			\doparttoc
			\dopartlof
			\dopartlot
			
			


		\title{\LaTeX 김대희 개조식 문장 작성 }
		\author{김대희}
		\date{\today}
		\maketitle




			



% 	==============================================================================  style


		\mdfdefinestyle	{code_document} {
						outerlinewidth		=1pt			,%
						innerlinewidth		=2pt			,%
						outerlinecolor		=blue!70!black	,%
						innerlinecolor		=white 			,%
						roundcorner			=4pt			,%
						skipabove			=1.0em 			,%
						skipbelow			=1.0em 			,%
						leftmargin			=0em			,%
						rightmargin			=0em			,%
						innertopmargin		=1em 			,%
						innerbottommargin 	=1em 			,%
						innerleftmargin		=1em 			,%
						innerrightmargin	=1em 			,%
						backgroundcolor		=gray!4			,%
						frametitlerule		=true 			,%
						frametitlerulecolor	=white			,%
						frametitlebackgroundcolor=gray		,%
						frametitleaboveskip=0.4em 			,%
						frametitlebelowskip=0.4em 			,%
						frametitlefontcolor=white 			,%
						}



					

% ===========================================================	part		=============
		\addtocontents{toc}{\protect\newpage}
		\part{List - 개조식 문서 작성}

		\parttoc
		\partlof
		\partlot

% 	================================================= chapter 	====================
%
%	-------------------------------------------------------------------------------
	\chapter{개조식 문서 작성}
	\minitoc				% Creating an actual minitoc

	\section{개조식 문서 작성}



	\section{모양 바꾸기}


	%	------------------------------------------------ code	
		\begin{mdframed}[style=code_document, frametitle={code}]
			\begin{verbatim}
		%		Vertical spacing
			\setlist[itemize]{topsep=0.0em}				% 상단의 여유치
			\setlist[itemize]{partopsep=0.0em}			% 
			\setlist[itemize]{parsep=0.0em}				% 
			\setlist[itemize]{itemsep=0.0em}			% 
			\setlist[itemize]{noitemsep}				% 
			
		%		Horizontal spacing
			\setlist[itemize]{labelwidth=1em}			%  라벨의 표시 폭
			\setlist[itemize]{leftmargin=8em}			%  본문 까지의 왼쪽 여백  - 4em
			\setlist[itemize]{labelsep=3em} 				%  본문에서 라벨까지의 거리 -  3em
			\setlist[itemize]{rightmargin=0em}			% 오른쪽 여백  - 4em
			\setlist[itemize]{itemindent=0em} 			% 점 내민 거리 label sep 과 같은면 점위치 까지 내민다
			\setlist[itemize]{listparindent=3em}			% 본문 드려쓰기 간격
	
	
			\setlist[itemize]{ topsep=0.0em, 			%  상단의 여유치
						partopsep=0.0em, 			%  
						parsep=0.0em, 
						itemsep=0.0em, 
						labelwidth=1em, 
						leftmargin=2.5em,
						labelsep=2em,			%  본문에서 라벨 까지의 거리
						rightmargin=0em,		% 오른쪽 여백  - 4em
						itemindent=0em, 		% 점 내민 거리 label sep 과 같은면 점위치 까지 내민다
						listparindent=0em}		% 본문 드려쓰기 간격
	
%			\begin{itemize}
			\end{verbatim}
		\end{mdframed}

	

	


%	------------------------------------------------------------------------------- itemize
	\chapter{itemize}

	% ==============================================================================	
	%	itemize
	% ==============================================================================	

	\section{itemize}

		% ----------------------------- itemize
			\begin{itemize}	[					%
						topsep=0.0em 			,%  상단의 여유치
						partopsep=0.0em 		,%  
						parsep=0.0em 			, 
						itemsep=0.0em 			, 
						leftmargin=5em 		,
						labelwidth=1em 			, 
						labelsep=4em 			,%  본문에서 라벨 까지의 거리
						rightmargin=0em 		,% 오른쪽 여백  - 4em
						itemindent=0em 			,% 점 내민 거리 label sep 과 같은면 점위치 까지 내민다
						listparindent=0em 		,% 본문 드려쓰기 간격
						]
			\item	1
			\item	2
			\item	3
			\end{itemize}

	%	------------------------------------------------ code	
		\begin{mdframed}[style=code_document, frametitle={code}]
			\begin{verbatim}
			\begin{itemize}	[					%
						topsep=0.0em 			,%  상단의 여유치
						partopsep=0.0em 		,%  
						parsep=0.0em 			, 
						itemsep=0.0em 			, 
						leftmargin=5em 		,
						labelwidth=1em 			, 
						labelsep=4em 			,%  본문에서 라벨 까지의 거리
						rightmargin=0em 		,% 오른쪽 여백  - 4em
						itemindent=0em 			,% 점 내민 거리 label sep 과 같은면 점위치 까지 내민다
						listparindent=0em 		,% 본문 드려쓰기 간격
						]
			\item	1
			\item	2
			\item	3
			\end{itemize}
			\end{verbatim}
		\end{mdframed}



			
			\begin{itemize}[topsep=0.0em,itemsep=0.0em]
			\item	
			\end{itemize}	


			[itemsep=-0.5em]
			[topsep=-1.0em,			]


			\begin{itemize}[	topsep=0.0em,itemsep=0.0em,
							leftmargin=4em, labelsep=3em ]
			\item	
			\end{itemize}	

			
		% ----------------------------- itemize
		\def\TStart{	\setlength{\fboxsep}{12pt}
					\begin{boxedminipage}[c]{1.0\linewidth}
					\begin{itemize}[topsep=0.0em,itemsep=0.0em]
					}
		\def\TEnd{	\end{itemize}	
					\end{boxedminipage}\\
					}
		\TStart
		\item 1 
		\item 2 
		\TEnd

		\setlength{\fboxsep}{12pt}
		\begin{boxedminipage}[c]{1.0\linewidth}
		\end{boxedminipage}\\



%	------------------------------------------------------------------------------- enumerate
	\chapter{enumerate}

	% ==============================================================================	
	%	enumerate
	% ==============================================================================	

	\section{enumerate}

		% ----------------------------- enumerate
			\begin{enumerate}
			\setlength\itemsep{-1.0em}
			\item	enumerate test 1
			\item	enumerate test 2
				\begin{enumerate}
				\setlength\itemsep{-1.0em}
				\item	enumerate test 1
				\item	enumerate test 2
					\begin{enumerate}
					\setlength\itemsep{-1.0em}
					\item	enumerate test 1
					\item	enumerate test 2
					\end{enumerate}
				\item	enumerate test 5
				\item	enumerate test 6
				\end{enumerate}
			\item	enumerate test 5
			\item	enumerate test 6
			\end{enumerate}
		


	%	------------------------------------------------ code	
		\begin{mdframed}[style=code_document, frametitle={code}]
			\begin{verbatim}
			\begin{enumerate}
			\setlength\itemsep{-1.0em}
			\item	enumerate test 1
			\item	enumerate test 2
				\begin{enumerate}
				\setlength\itemsep{-1.0em}
				\item	enumerate test 1
				\item	enumerate test 2
					\begin{enumerate}
					\setlength\itemsep{-1.0em}
					\item	enumerate test 1
					\item	enumerate test 2
					\end{enumerate}
				\item	enumerate test 5
				\item	enumerate test 6
				\end{enumerate}
			\item	enumerate test 5
			\item	enumerate test 6
			\end{enumerate}
			\end{verbatim}
		\end{mdframed}


	\section{enumerate : label}
		% -------------------------------------
		% 	arabic*
		% -------------------------------------
			\begin{enumerate}[ label=\arabic*)]
			\setlength\topsep{0.0em}
			\setlength\itemsep{-1.0em}
			\item	enumerate test 1
			\item	enumerate test 2
			\item	enumerate test 3
			\end{enumerate}

	%	------------------------------------------------ code	
		\begin{mdframed}[style=code_document, frametitle={code}]
			\begin{verbatim}
			\begin{enumerate}[ label=\arabic*)]
			\setlength\topsep{0.0em}
			\setlength\itemsep{-1.0em}
			\item	enumerate test 1
			\item	enumerate test 2
			\item	enumerate test 3
			\end{enumerate}
			\end{verbatim}
		\end{mdframed}


	\section{enumerate}


		\paragraph{enumerate : defalult}
	% ----------------------------- enumerate
			\begin{enumerate}
			\setlength\topsep{0.0em}
			\setlength\itemsep{-1.0em}
			\setlength\labelwidth{0.0cm}
			\setlength\labelsep{1cm}
	
			\item	enumerate test 1
			\item	enumerate test 2
			\item	enumerate test 3
			\item	enumerate test 4
			\item	enumerate test 5
			\item	enumerate test 6
			\end{enumerate}

		\paragraph{enumerate : leftmargin=2cm}
	% ----------------------------- enumerate
			\begin{enumerate}[ leftmargin=2cm ]
			\setlength\topsep{0.0em}
			\setlength\itemsep{-1.0em}
			\setlength\labelwidth{0.0cm}
			\setlength\labelsep{1cm}
	
			\item	enumerate test 1
			\item	enumerate test 2
			\item	enumerate test 3
			\item	enumerate test 4
			\item	enumerate test 5
			\item	enumerate test 6
			\end{enumerate}


		\paragraph{enumerate : leftmargin=4cm}
		% ----------------------------- enumerate
			\begin{enumerate}[leftmargin=4cm]
			\setlength\topsep{0.0em}
			\setlength\itemsep{-1.0em}
			\setlength\leftmargin{0cm}
			\setlength\rightmargin{6cm}
	
			\setlength\labelsep  {1.0cm}
			\setlength\labelwidth{0.0cm}
			\setlength\listparindent{0.0em}
	
			\item	enumerate test 1
			\item	enumerate test 2
			\item	enumerate test 3
			\item	enumerate test 4
			\item	enumerate test 5
			\item	enumerate test 6
			\end{enumerate}


		

	\section{enumerate}


		\paragraph{ leftmargin\{8cm\}, rightmargin\{10cm\} } 
		% ----------------------------- enumerate
			\begin{enumerate}
			\setlength\leftmargin{8cm}
			\setlength\rightmargin{10cm}
			\item	보강토체를 따른 활동(성토체내의 흙과 흙 사이의 내부마찰각) 
			\item	기초지반을 따른 활동(성토체흙과 기초지반흙과의 마찰각) 
			\item	최하단 토목섬유 보강재와 흙 사이의 경계면을 따른 활동 
			\end{enumerate}
	
		\paragraph{ leftmargin\{4cm\}, rightmargin\{4cm\} } 
		% ----------------------------- enumerate
			\begin{enumerate}	[
							leftmargin=4cm, 
							rightmargin=4cm
							]
			\item	보강토체를 따른 활동(성토체내의 흙과 흙 사이의 내부마찰각) 
			\item	기초지반을 따른 활동(성토체흙과 기초지반흙과의 마찰각) 
			\item	최하단 토목섬유 보강재와 흙 사이의 경계면을 따른 활동 
			\end{enumerate}




	\section{dingautolist}

		% ----------------------------- enumerate
			\begin{dingautolist}{172}
			\setlength\itemsep{-1.0em}
			\setlength\leftmargin{8cm}
			\setlength\rightmargin{10cm}
			\item	보강토체를 따른 활동(성토체내의 흙과 흙 사이의 내부마찰각) 
			\item	보강토체를 따른 활동(성토체내의 흙과 흙 사이의 내부마찰각) 
			\item	보강토체를 따른 활동(성토체내의 흙과 흙 사이의 내부마찰각) 
			\item	보강토체를 따른 활동(성토체내의 흙과 흙 사이의 내부마찰각) 
			\item	보강토체를 따른 활동(성토체내의 흙과 흙 사이의 내부마찰각) 
			\item	보강토체를 따른 활동(성토체내의 흙과 흙 사이의 내부마찰각) 
			\item	보강토체를 따른 활동(성토체내의 흙과 흙 사이의 내부마찰각) 
			\item	보강토체를 따른 활동(성토체내의 흙과 흙 사이의 내부마찰각) 
			\item	보강토체를 따른 활동(성토체내의 흙과 흙 사이의 내부마찰각) 
			\item	보강토체를 따른 활동(성토체내의 흙과 흙 사이의 내부마찰각) 
			\item	보강토체를 따른 활동(성토체내의 흙과 흙 사이의 내부마찰각) 
			\item	보강토체를 따른 활동(성토체내의 흙과 흙 사이의 내부마찰각) 
			\item	기초지반을 따른 활동(성토체흙과 기초지반흙과의 마찰각) 
			\item	최하단 토목섬유 보강재와 흙 사이의 경계면을 따른 활동 
			\item	기초지반을 따른 활동(성토체흙과 기초지반흙과의 마찰각) 
			\item	최하단 토목섬유 보강재와 흙 사이의 경계면을 따른 활동 
			\item	기초지반을 따른 활동(성토체흙과 기초지반흙과의 마찰각) 
			\item	최하단 토목섬유 보강재와 흙 사이의 경계면을 따른 활동 
			\item	기초지반을 따른 활동(성토체흙과 기초지반흙과의 마찰각) 
			\item	최하단 토목섬유 보강재와 흙 사이의 경계면을 따른 활동 
			\item	기초지반을 따른 활동(성토체흙과 기초지반흙과의 마찰각) 
			\item	최하단 토목섬유 보강재와 흙 사이의 경계면을 따른 활동 
			\item	기초지반을 따른 활동(성토체흙과 기초지반흙과의 마찰각) 
			\item	최하단 토목섬유 보강재와 흙 사이의 경계면을 따른 활동 
			\item	기초지반을 따른 활동(성토체흙과 기초지반흙과의 마찰각) 
			\item	최하단 토목섬유 보강재와 흙 사이의 경계면을 따른 활동 
			\end{dingautolist}


	\section{사용자 정의}

		% ----------------------------- 사용자 정의
			\begin{enumerate}[label=\bfseries Exercise \arabic*:]
			\setlength\itemsep{1em}
			 \item 5 + 7 = 12
			 \item 9 + 1 = 10
			 \item 2 $\times$ 2 = 4
			\end{enumerate}


	%	------------------------------------------------ code	
		\begin{mdframed}[style=code_document, frametitle={code}]
			\begin{verbatim}
			\begin{enumerate}[label=\bfseries Exercise \arabic*:]
			\setlength\itemsep{1em}
			 \item 5 + 7 = 12
			 \item 9 + 1 = 10
			 \item 2 $\times$ 2 = 4
			\end{enumerate}
			\end{verbatim}
		\end{mdframed}



%	------------------------------------------------------------------------------- description
	\chapter{description}


%	-----------------------------------------------------------  section  
	\section{description}


			left margin은 style=sameline에서만 적용된다.  \\

			labelindent=0pt: to have a flush left margin



%	-----------------------------------------------------------  section  
	\section{description : defalult - align = left}
			% ----------------------------- description
			\paragraph{ align = left }
				\begin{description}[
						topsep=0.0em 			,%  상단의 여유치
						partopsep=0.0em 		,%  
						parsep=0.0em 			, 
						itemsep=0.0em 			, 
						leftmargin=10em 		,
						labelwidth=10em 			, 
						labelsep=0em 			,%  본문에서 라벨 까지의 거리
						rightmargin=0em 		,% 오른쪽 여백  - 4em
						itemindent=0em 			,% 점 내민 거리 label sep 과 같은면 점위치 까지 내민다
						listparindent=0em 		,% 본문 드려쓰기 간격
						align=left				,
						]
				\item	[description 1]		description test 1description test 1description test 1description test 1description test 1
				\item	[description]		description test 2
				\item	[descrip]			description test 3
				\item	[descrip]			description test 4
				\item	[desc]			description test 5
				\item	[d]				description test 6
				\end{description}
	
	%	------------------------------------------------ code	
		\begin{mdframed}[style=code_document, frametitle={code}]
			\begin{verbatim}
				\begin{description}[
						topsep=0.0em 			,%  상단의 여유치
						partopsep=0.0em 		,%  
						parsep=0.0em 			, 
						itemsep=0.0em 			, 
						leftmargin=10em 		,
						labelwidth=10em 			, 
						labelsep=0em 			,%  본문에서 라벨 까지의 거리
						rightmargin=0em 		,% 오른쪽 여백  - 4em
						itemindent=0em 			,% 점 내민 거리 label sep 과 같은면 점위치 까지 내민다
						listparindent=0em 		,% 본문 드려쓰기 간격
						align=left				,
						]
				\item	[description 1]		description test 1description test 1description test 1description test 1description test 1
				\item	[description]		description test 2
				\item	[descrip]			description test 3
				\item	[descrip]			description test 4
				\item	[desc]			description test 5
				\item	[d]				description test 6
				\end{description}
			\end{verbatim}
		\end{mdframed}


%	-----------------------------------------------------------  section  
	\section{description : defalult - align = right}
			\paragraph{ align = right }
				\begin{description}[
						topsep=0.0em 			,%  상단의 여유치
						partopsep=0.0em 		,%  
						parsep=0.0em 			, 
						itemsep=0.0em 			, 
						leftmargin=8em 		,
						labelwidth=8em 		, 
						labelsep=2em 			,%  본문에서 라벨 까지의 거리
						rightmargin=0em 		,% 오른쪽 여백  - 4em
						itemindent=0em 			,% 점 내민 거리 label sep 과 같은면 점위치 까지 내민다
						listparindent=0em 		,% 본문 드려쓰기 간격
						align=right				,
						]
				\item	[description 1]	description test 1
				\item	[description]		description test 2
				\item	[descrip]			description test 3
				\item	[descrip]			description test 4
				\item	[desc]			description test 5
				\item	[d]				description test 6
				\end{description}

	%	------------------------------------------------ code	
		\begin{mdframed}[style=code_document, frametitle={code}]
			\begin{verbatim}
				\begin{description}[
						topsep=0.0em 			,%  상단의 여유치
						partopsep=0.0em 		,%  
						parsep=0.0em 			, 
						itemsep=0.0em 			, 
						leftmargin=8em 		,
						labelwidth=8em 		, 
						labelsep=2em 			,%  본문에서 라벨 까지의 거리
						rightmargin=0em 		,% 오른쪽 여백  - 4em
						itemindent=0em 			,% 점 내민 거리 label sep 과 같은면 점위치 까지 내민다
						listparindent=0em 		,% 본문 드려쓰기 간격
						align=right				,
						]
				\item	[description 1]	description test 1
				\item	[description]		description test 2
				\item	[descrip]			description test 3
				\item	[descrip]			description test 4
				\item	[desc]			description test 5
				\item	[d]				description test 6
				\end{description}
			\end{verbatim}
		\end{mdframed}





%	-----------------------------------------------------------  section  
	\section{description : style=standard : align = left}
			% ----------------------------- description
			\paragraph{ align = left }
			\begin{description}	[
							style=standard, 
							align=left,
							itemsep=0.0em 			,% 
							labelsep=2em 			,%
							]
			\setlength\labelsep{2em}
			\item	[description 1]	description test 1
			\item	[description]		description test 2
			\item	[descrip]			description test 3
			\item	[descrip]			description test 4
			\item	[desc]			description test 5
			\item	[d]				description test 6
			\end{description}


	%	------------------------------------------------ code	
		\begin{mdframed}[style=code_document, frametitle={code}]
			\begin{verbatim}
			\begin{description}	[
							style=standard, 
							align=left,
							itemsep=0.0em 			,% 
							labelsep=2em 			,%
							]
			\setlength\labelsep{2em}
			\item	[description 1]	description test 1
			\item	[description]		description test 2
			\item	[descrip]			description test 3
			\item	[descrip]			description test 4
			\item	[desc]			description test 5
			\item	[d]				description test 6
			\end{description}
			\end{verbatim}
		\end{mdframed}



%	-----------------------------------------------------------  section  
	\section{description : style=standard : align = right}

			\paragraph{ align = right }
			\begin{description}	[
							style=standard, 
							align=right,
							labelsep=2em,
							leftmargin=8em,
							]
			\item	[description 1]	description test 1description test 1description test 1description test 1description test 1description test 1description test 1description test 1
			\item	[description]		description test 2
			\item	[descrip]			description test 3
			\item	[descrip]			description test 4
			\end{description}

	%	------------------------------------------------ code	
		\begin{mdframed}[style=code_document, frametitle={code}]
			\begin{verbatim}
			\begin{description}	[
							style=standard, 
							align=right,
							labelsep=2em,
							leftmargin=8em,
							]
			\item	[description 1]	description test 1description test 1description test 1description test 1description test 1description test 1description test 1description test 1
			\item	[description]		description test 2
			\item	[descrip]			description test 3
			\item	[descrip]			description test 4
			\end{description}
			\end{verbatim}
		\end{mdframed}


%	-----------------------------------------------------------  section  
	\section{description : style=standard : align = right}

			\paragraph{ align = right }
			\begin{description}	[
							style=standard, 
							align=right,
							labelwidth=8em,
							labelsep=2em,
							leftmargin=10em,
							]
			\item	[description 1]	description test 1description test 1description test 1description test 1description test 1description test 1description test 1description test 1
			\item	[description]		description test 2
			\item	[descrip]			description test 3
			\item	[descrip]			description test 4
			\end{description}
			
			labelwidth=8em + labelsep=2em = leftmargin=10em

	%	------------------------------------------------ code	
		\begin{mdframed}[style=code_document, frametitle={code}]
			\begin{verbatim}
			\begin{description}	[
							style=standard, 
							align=right,
							labelwidth=8em,
							labelsep=2em,
							leftmargin=10em,
							]
			\item	[description 1]	description test 1description test 1description test 1description test 1description test 1description test 1description test 1description test 1
			\item	[description]		description test 2
			\item	[descrip]			description test 3
			\item	[descrip]			description test 4
			\end{description}
			\end{verbatim}
		\end{mdframed}


%	-----------------------------------------------------------  section  
	\section{description : style=standard : align = parleft}

			\paragraph{ align = parleft }
			\begin{description}	[
							style=standard, 
							align=parleft,
							leftmargin=10em,
							labelsep=1em
							]
			\item	[description1]		description test 1 description test 1 description test 1 description test 1 description test 1 description test 1
			\item	[description]		description test 2
			\item	[descrip]			description test 3
			\end{description}


	%	------------------------------------------------ code	
		\begin{mdframed}[style=code_document, frametitle={code}]
			\begin{verbatim}
			\begin{description}	[ style=standard, align=parleft, 
								leftmargin=10em, labelsep=1em ]
			\end{description}
			\end{verbatim}
		\end{mdframed}

			\paragraph{ align = parleft }
			\begin{description}	[
							style=standard, 
							align=parleft,
							leftmargin=10em,
							labelsep=1em,
							labelwidth=5em
							]
			\item	[description1]		description test 1 description test 1 description test 1 description test 1 description test 1 description test 1
			\item	[description]		description test 2
			\item	[descrip]			description test 3
			\end{description}

	%	------------------------------------------------ code	
		\begin{mdframed}[style=code_document, frametitle={code}]
			\begin{verbatim}
			\begin{description}	[style=standard, align=parleft,
							leftmargin=10em, labelsep=1em, labelwidth=5em ]
			\end{description}
			\end{verbatim}
		\end{mdframed}


%	-----------------------------------------------------------  section  
	\section{description : unboxed}
			% ----------------------------- description
			\begin{description}[style=unboxed]
			\setlength\topsep{0.0em}
			\setlength\itemsep{-1.0em}
			\setlength\leftmargin{12cm}
			\setlength\labelwidth{2cm}
			\setlength\labelsep{2em}


			\item	[description 1]	description test 1
			\item	[description]		description test 2
			\item	[descrip]			description test 3
			\item	[descrip]			description test 4
			\item	[desc]			description test 5
			\item	[d]				description test 6
			\end{description}


	\section{description : nextline}
			% ----------------------------- description
			\begin{description}[style=nextline, align=left]
			\setlength\topsep{0.0em}
			\setlength\itemsep{-1.0em}

			\item	[description 1]	description test 1
			\item	[description]		description test 2
			\item	[descrip]			description test 3
			\item	[descrip]			description test 4
			\item	[desc]			description test 5
			\item	[d]				description test 6
			\end{description}


			% ----------------------------- description
			\begin{description}[style=nextline]
			\setlength\topsep{0.0em}
			\setlength\itemsep{-1.0em}

			\item	[description 1]	description test 1
			\item	[description]		description test 2
			\item	[descrip]			description test 3
			\item	[descrip]			description test 4
			\item	[desc]			description test 5
			\item	[d]				description test 6
			\end{description}




			% ----------------------------- description
			%	sameline
			% ----------------------------- description

	\section{description : sameline}
			% ----------------------------- description
			\begin{description}[style=sameline]
			\setlength\topsep{0.0em}
			\setlength\itemsep{-1.0em}
			\item	[description 1]	description test 1
			\item	[description]		description test 2
			\item	[descrip]			description test 3
			\item	[descrip]			description test 4
			\item	[desc]			description test 5
			\item	[d]				description test 6
			\end{description}


			\begin{description}[style=sameline, leftmargin=2cm]
			\setlength\topsep{0.0em}
			\setlength\itemsep{-0.5em}
%\singlespacing
%\onehalfspacing
%\doublespacing

			\item	[description 1]	description test 1
			\item	[description]		description test 2
			\item	[descrip]			description test 3
			\item	[descrip]			description test 4
			\item	[desc]			description test 5
			\item	[d]				description test 6
			\item[$b$] Taxa agregada de bits alcançável para o sistema
			\item[$H(f)$] Espectro do canal
			\item[$H_k$] Ganho do $k$-ésimo subcanal
			\item[$P_x$] Potência total de transmissão
			\item[$s_k$] Densidade espectral de potência do sinal no $k$-ésimo subcanal
			\item[$Sx(f)$] Densidade espectral de potência do sinal na frequência contínua
			\item[$Sn(f)$] Densidade espectral de potência do AWGN na frequência contínua
			\item[1]	보강토체를 따른 활동(성토체내의 흙과 흙 사이의 내부마찰각) 
					보강토체를 따른 활동(성토체내의 흙과 흙 사이의 내부마찰각) 
			\item[2]	기초지반을 따른 활동(성토체흙과 기초지반흙과의 마찰각) 
			\item[3]	최하단 토목섬유 보강재와 흙 사이의 경계면을 따른 활동 

			\end{description}

		
	\section{description : align=left}
		% ----------------------------- description
		\begin{description}[labelsep=3em, align=left]
		\setlength\topsep{0.0em}
		\setlength\itemsep{-1.0em}

		\item[$b$] Taxa agregada de bits alcançável para o sistema
		\item[$H(f)$] Espectro do canal
		\item[$H_k$] Ganho do $k$-ésimo subcanal
		\item[$P_x$] Potência total de transmissão
		\item[$s_k$] Densidade espectral de potência do sinal no $k$-ésimo subcanal
		\item[$Sx(f)$] Densidade espectral de potência do sinal na frequência contínua
		\item[$Sn(f)$] Densidade espectral de potência do AWGN na frequência contínua
		\end{description}




	\section{description : align=right}
		% ----------------------------- description
		\begin{description}[labelsep=3em, align=right]
		\setlength\topsep{0.0em}
		\setlength\itemsep{-1.0em}

		\item[$b$] Taxa agregada de bits alcançável para o sistema
		\item[$Sn(f)$] Densidade espectral de potência do AWGN na frequência contínua
		\item[$\text{SNR}_k$] Razão sinal-ruído no subcanal $k$
		\item[$\mathbf{X}$] Vetor correspondente ao símbolo DMT
		\item[$\mathbf{X}_+$] Vetor de subsímbolos dos tons positivos do símbolo DMT.
		\item[$X_k$] Subsímbolo no $k$-ésimo tom do símbolo DMT
		\item[$\Gamma$] \textsl[Gap] de SNR a capacidade
		\item[$\Delta f$] Largura de banda do subcanal (espaçamento tonal).
		\item[$\sigma_k$] Densidade espectral de potência do AWGN no $k$-ésimo subcanal
		\end{description}




%	------------------------------------------------------------------------------- parlist
	\chapter{parlist}


	% ==============================================================================	
	%	paralist
	% ==============================================================================	
	%

\section{paralist}





%	------------------------------------------------------------------------------- tabbing
	\chapter{tabbing}




	% ==============================================================================	
	%	TAB
	% ==============================================================================	
	%

\section{tabbing}


		\begin{boxedminipage}[c,fboxrule=10pt,fboxsep=12pt]{1.0\linewidth}
		\begin{itemize}
		\item[]	\textbackslash begin \{tabbing\}
		\item[]	text \textbackslash = more text \textbackslash = still more text \textbackslash = last text \textbackslash\textbackslash 
		\item[]	\textbackslash end \{tabbing\}
		\end{itemize}
		\end{boxedminipage} \\


		\paragraph{set the tab position}


		\begin{tabbing}
		\hspace{2cm}	\= \hspace{2cm}	\= \hspace{2cm}	\\
		second row 	\> text-- 	\> more \\
		second row 	\> text-- 	\> more \\
		\end{tabbing}

		\begin{tabbing}
		\hspace{2cm}	\=\hspace{2cm}	\=\hspace{2cm}\\
		second row 	\>  		\> more \\
		second row 	\>  		\> more \\
		\end{tabbing}

		\paragraph{Tabbing commands}
		\begin{description}[itemsep=-0.5em,style=sameline]
		\item [\textbackslash =] set tab
		\item [\textbackslash $>$] advance to next tab stop
		\item [\textbackslash $<$]
		\item [\textbackslash $+$] indent; move margin right
		\item [\textbackslash $-$] unindent; move margin left
		\item [\textbackslash ']
		\item [\textbackslash `]
		\item [\textbackslash \textbackslash] end of line; newline
		\item [\textbackslash kill] ignore preceding text; use only for spacing
		\end{description}




%	------------------------------------------------------------------------------- tabenum
	\chapter{tab enum}

	% ==============================================================================	
	%	문제 풀이의 보기 항목
	%	tablists package
	% ==============================================================================	

\section{tabenum}



		\begin{tabenum}[\bfseries1)]%
		\tabenumitem 		$z=\displaystyle\frac xy$
		\tabenumitem 		$z=\displaystyle\frac xy$
		\tabenumitem 		$z=\displaystyle\frac xy$
		\tabenumitem 		$z=\displaystyle\frac xy$
		\tabenumitem 		$z=\displaystyle\frac xy$
		\tabenumitem 		$z=\displaystyle\frac xy$
		\tabenumitem 		$z=\displaystyle\frac xy$\\
		\tabenumitem 		$z=\displaystyle\frac xy$ 
		\skipitem
		\tabenumitem 		$6 z=\displaystyle\frac xy$\\	
		\item			$ item  z=\displaystyle\frac xy$
		\noitem			$ noitem 2^x=9$
		\item			$2^x=9 noitem$\cr
		\item			$2^x=9$\cr
		\item			$2^x=9$\cr
		\item			$2^x=9$\cr
		\tabenumitem		$3^{2x+3}=16 $
		\tabenumitem		$z=2x^2+4y^2$\par
		\tabenumitem		$u=\sqrt{x^2+y^2+z^2}$
		\tabenumitem		$v=gt+\displaystyle\frac{g}{4}t$;\\[1ex]
		\tabenumitem		$v=gt+\frac{g}{4}t$;\\[1ex]
%		\tabenumitem		$u=2ˆ{5x-3y+z}$;
%		\tabenumitem		$w=(v+7)ˆ2+(u-3)ˆ2$;
%		\tabenumitem		$5ˆx=\displaystyle\frac{4}{3} ;$
%		\tabenumitem		$z=(x+1)ˆ2+yˆ2$;\\*
%		\tabenumitem		$2+5+8+ \ldots +(3n+2)=155$, $n\in \mathrm{N};$
%		\tabenumitem		$t=5uˆ2+8vˆ2$;
		\end{tabenum}











\end{document}

% -------------------------------------------------------------------------------------------------------------------------------------
